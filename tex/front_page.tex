%------------------------------------%
%                                    %
% Fichier devoir mth1101                   %
% 													         %
% Date: 2 juillet 2007      				 %
%                                    %
% Auteur: Jean Guerin                %
%                                    %
%------------------------------------%

\documentclass[10pt]{article}

\usepackage[utf8]{inputenc}
\usepackage[english,french]{babel}
\usepackage{fancybox}

% commandes pour personnaliser la page de couverture
\newcommand{\cours}{MTH1102 / MTH1102H - Calcul II}
\newcommand{\session}{Hiver 2022}
\newcommand{\devoir}{Devoir 5}

% nom et matricule sont redéfinis dans le fichier data.tex
% Sera utilisé par le programme python
\newcommand{\nom}{}
\newcommand{\matricule}{}

% mise en page %
\setlength{\parindent}{0cm}
\renewcommand{\baselinestretch}{1}
\usepackage{geometry}
\geometry{letterpaper, tmargin=2cm, bmargin=2.0cm, lmargin=2.25cm, rmargin=2.25cm, headsep=1.0cm}

\usepackage[usenames]{color}
\definecolor{gris1}{gray}{0.75}

\newcommand{\encadre}[1]{
\setlength\fboxsep{5mm}\setlength\fboxrule{1pt}
\begin{center}
\fcolorbox{black}{gris1}{
\begin{minipage}{0.94\textwidth}{#1}\end{minipage}}
\end{center}}

% encadre blanc
\newcommand{\boite}[1]{
\setlength\fboxsep{5mm}\setlength\fboxrule{1pt}
\begin{center}
\fcolorbox{black}{white}{
\begin{minipage}{0.5\textwidth}{#1}\end{minipage}}
\end{center}}


\begin{document}

\renewcommand{\nom}{George Gabriel Stokes}
\renewcommand{\matricule}{17111887}

\thispagestyle{empty}

\encadre{
\begin{center}
\bf
{\Large \scshape 
Polytechnique Montr\'eal
\\
D\'epartement de Math\'ematiques et de G\'enie Industriel
}
\\
{\Huge
\

\cours
\\
\session

\

\devoir

}
\end{center}
}

{
\centering

\vfill

\fcolorbox{black}{white}{
\begin{minipage}{0.94\linewidth}

\vspace{5mm}

{\Large
{\bf Nom et Prénom: }\nom

\vspace{8mm}

{\bf Matricule : }\matricule
}

%\vspace{8mm}
%
%{\bf \Large Signature: }\rule[-1mm]{126mm}{0.6pt}
%
%
%\vspace{5mm}

\end{minipage}}

\vfill

{
\renewcommand{\arraystretch}{1.5}
\begin{center}
\begin{tabular}{|c|c|c||c|} \hline
{\bf \Large Question}							& {\bf \Large Autres}			& 	{\bf \Large Bonus}	&  \\ 
{\bf \Large corrig\'ee}						& {\bf \Large  questions}	&		{\bf \Large \LaTeX}	& {\bf \Large Total} \\ \hline
\hspace{20mm}			{\Huge \strut}	& \hspace{20mm}						&		\hspace{20mm}				&\hspace{20mm} \\
\hspace{20mm}			{\Huge \strut}	& \hspace{20mm}						&		\hspace{20mm} 			& \hspace{20mm} {\Large /10} \\ \hline
\end{tabular}
\end{center}
}


\vfill
}

\end{document}